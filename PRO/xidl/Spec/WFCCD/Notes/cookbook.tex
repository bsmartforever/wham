\documentclass[11pt,letterpaper,dvips]{article}
 
 
\usepackage{latexsym}
\usepackage{fancybox}
\usepackage{graphicx}
\usepackage{amssymb}
\usepackage{color}
%\usepackage{amstex}
\usepackage{ulem}
\usepackage{float}
 
\pretolerance=10000
\textwidth=7.0in
\textheight=10.0in
\voffset = -0.3in
\topmargin=0.0in
\headheight=0.00in
\hoffset = -0.3in
\headsep=0.00in
\oddsidemargin=0in
\evensidemargin=0in
\parindent=2em
\parskip=1.5ex
 
\def\rexp{{\rm e}}
\def\katzt{\ket{\alpha, t_0=0; t}}
\def\katt{\ket{\alpha, t_0; t}}
\def\usc{\mathcal{U}}
\def\tsc{\mathcal{T}}
\special{papersize=8.5in,11in}
\newcommand{\nid}{\noindent}
\newcommand{\ew}{W_\lambda}
\def\aap{A \& A}
\def\aj{AJ}
\def\apj{ApJ}
\def\apjl{ApJL}
\def\apjs{ApJS}
\def\apjsupp{ApJS}
\def\araa{ARAA}
\def\mnras{MNRAS}
\def\nat{Nature}
\def\pasp{PASP}
\def\hangpara{\par\hangindent 25pt\noindent}
\def\Lya{Ly$\alpha$}
\def\lya{Ly$\alpha$}
\def\Lyb{Ly$\beta$ }
\def\lyb{Ly$\beta$ }
\def\Lyg{Ly$\gamma$ }
\def\Lyd{Ly$\delta$ }
\def\etal{et al. }
\def\kms{km~s$^{-1}$ }
\def\mkms{{\rm \; km\,s\,^{-1}}}
\def\bmkms{{\rm\bf \; km\,s\,^{-1}}}
\def\L{$\lambda$}
\def\s-1{s$^{-1}$}
\def\Hz-1{Hz$^{-1}$}
\def\pohf{\rm p_\ohf}
\def\prhf{\rm p_\rhf}
\def\ohf{\frac{1}{2}}
\def\nohf{\frac{-1}{2}}
\def\rhf{\frac{3}{2}}
\def\sohf{\rm s_\ohf}
\def\esp{\Delta E_{2s, 2p}}
\def\chal{\frac{\Delta \alpha}{\alpha}}
\def\ylab{y_{\rm lab}}
\def\yaY{y \; {\rm and} \; Y}
\def\dadz{\frac{1}{\alpha} |\frac{{\rm d} \alpha}{{\rm d} z}|}
\def\abdadz{\frac{1}{\alpha} \left|\frac{{\rm d} \alpha}{{\rm d} z}\right|}
\def\abdadt{\frac{1}{\alpha} \left|\frac{{\rm d} \alpha}{{\rm d} t}\right|}
\def\sci#1{{\; \times \; 10^{#1}}}
\def\yold{y_{\rm old}}
\def\und#1{{\rm \underline{#1}}}
\def\dlmb{\Delta \lambda}
\def\abchal{| \chal |}
\def\lbar{\bar\lambda}
\def\elc{\rm e^-}
\def\qnn#1{\eqno {\rm {#1}}}
\def\l1l2{\lambda_2 \; {\rm and} \; \lambda_1}
\def\id{\indent}
\def\wid{\id\id}
\def\rid{\id\id\id}
\def\uid{\id\id\id\id}
\def\imp{\, \Rightarrow \>}
\def\di{\partial}
\def\ba{{\bf a}}
\def\bx{{\bf x}}
\def\bv{{\bf v}}
\def\bu{{\bf u}}
\def\bq{{\bf q}}
\def\br{{\bf r}}
\def\bs{{\bf s}}
\def\bp{{\bf p}}
\def\bj{{\bf j}}
\def\bk{{\bf k}}
\def\bm{{\bf m}}
\def\bn{{\bf n}}
\def\bd{{\bf d}}
\def\bg{{\bf g}}
\def\by{{\bf y}}
\def\bB{{\bf B}}
\def\bE{{\bf E}}
\def\bD{{\bf D}}
\def\bH{{\bf H}}
\def\bA{{\bf A}}
\def\bI{{\bf I}}
\def\bJ{{\bf J}}
\def\bK{{\bf K}}
\def\bL{{\bf L}}
\def\bS{{\bf S}}
\def\bF{{\bf F}}
\def\bN{{\bf N}}
\def\bP{{\bf P}}
\def\bR{{\bf R}}
\def\bU{{\bf U}}
\def\bM{{\bf M}}
\def\bY{{\bf Y}}
\def\bOm{{\bf \Omega}}
\def\bom{{\bf \omega}}
\def\bell{{\bf \ell}}
\def\di{\partial}
\def\ddi#1#2{\frac{\di #1}{\di #2}}
\def\gmm{\sqrt{1 - \frac{v^2}{c^2}}}
\def\eng{\varepsilon}
\def\dd#1#2{\frac{d #1}{d #2}}
\def\dt#1{\frac{d #1}{dt}}
\def\ddt{\frac{d}{d t}}
\def\ddit{\frac{\di}{\di t}}
\def\arr{\to \quad}
\def\smm{\sum\limits}
\def\intl{\int\limits}
\def\ointl{\oint\limits}
\def\liml{\lim\limits}
\def\prodl{\prod\limits}
\def\grd{{\bf \nabla}}
\def\crl{{\bf \grd} \times}
\def\dv{{\bf \grd} \cdot}
\def\ec{\frac{e}{c} \,}
\def\oc{\frac{1}{c}}
\def\xht{{\; \rm \hat x}}
\def\yht{{\; \rm \hat y}}
\def\zht{{\; \rm \hat z}}
\def\rht{{\; \rm \hat r}}
\def\pht{{\; \rm \hat \phi}}
\def\iht{{\; \rm \hat i}}
\def\jht{{\; \rm \hat j}}
\def\kht{{\; \rm \hat k}}
\def\theht{{\; \rm \hat \theta}}
\def\nht{{\; \rm \hat n}}
\def\oht{{\; \rm \hat 1}}
\def\tht{{\; \rm \hat 2}}
\def\vht{{\; \rm \hat v}}
\def\thht{{\; \rm \hat 3}}
\def\Rht{{\; \rm \hat R}}
\def\cyc{\Omega_c}
\def\dl{{d {\bf \ell}}}
\def\da{{\rm d {\bf a}}}
\def\ds{{\rm d {\bf s}}}
\def\smt{{\smll \rm T}}
\def\opi#1{\frac{1}{#1 \pi}}
\def\strs{\underline {\underline {\rm T}}}
\def\dtr{\, d^3 r}
\def\dftr{d^3 r \,}
\def\dtp{\, d^3 p}
\def\dftp{d^3 p \,}
\def\abrr{|\br - \br'|}
\def\rr{(\br - \br')}
\def\orr#1{\frac{#1}{\abrr}}
\def\trr#1{\frac{#1}{\abrr^3}}
\def\drr{\delta \rr}
\def\Grn{G(\br, \br')}
\def\Ylm{Y_{\ell m}}
\def\snth{\sin\theta}
\def\csth{\cos\theta}
\def\invr#1{\frac{1}{#1}}
\def\fpc{\frac{4 \pi}{c}}
\def\lgnl{P_\ell (\csth)}
\def\lgnn#1{{P_{#1} (\csth)}}
\def\thph{\theta, \phi}
\def\eht1{{\rm \hat e_1}}
\def\ehtb{{\rm \hat e_2}}
\def\iot{i \omega t}
\def\kr{\bk \cdot \br}
\def\ele{{\rm e^-}}
\def\iwc{\frac{i \omega}{c}}
\def\iwt{i \omega t}
\def\plr{{\rm \hat \epsilon}}
\def\skp{\smallskip}
\def\mkp{\medskip}
\def\bkp{\bigskip}
\def\elc{\rm e^{-s}}
\def\bmx{\left[\matrix}
\def\Det{{\rm Det \,}}
\def\bra#1{{<\negmedspace#1|}}
\def\ket#1{{|#1\negmedspace>}}
\def\kalp{\ket{\alpha}}
\def\kbet{\ket{\beta}}
\def\balp{\bra{\alpha}}
\def\bbet{\bra{\beta}}
\def\inpr#1#2
\def\otpr#1#2{{|#1><#2|}}
\def\dgg{\dagger}
\def\dual{\Leftrightarrow}
\def\hbm{\frac{\hbar^2}{2 m}}
\def\hbu{\frac{\hbar^2}{2 \mu}}
\def\mbh{\frac{2 m}{\hbar^2}}
\def\ih{\frac{i}{\hbar}}
\def\kjm{\ket{j m}}
\def\bjm{\bra{j m}}
\def\bjmp{\bra{j' m'}}
\def\ylm{Y_{\ell m}}
\def\pof#1{(\frac{#1}{2})}
\def\tqk{T_q^k}
\def\ptm{\frac{p^2}{2 m}}
\def\psn{\psi_n}
\def\tph{\frac{2 \pi}{\hbar}}
\def\plr{{\hat \epsilon}}
\def\Veff{V_{\rm eff}}
\def\rhs{\rho^{(s)}}
\def\rhis{\rho_i^{(s)}}
\def\tilH{\tilde H}
\def\tpi{2 \pi i}
\def\prin{\intl_P}
\def\ltk{\left [ \,}
\def\ltp{\left ( \,}
\def\ltb{\left \{ \,}
\def\rtk{\, \right  ] }
\def\rtp{\, \right  ) }
\def\rtb{\, \right \} }
\def\dc{\tilde d}
\def\bskp{\bigskip}
\def\nV{\Omega}
\def\d3x{d^3 x}
\def\mtI{\b{\b I}}
\def\cmma{\;\;\; ,}
\def\perd{\;\;\; .}
\def\delv{\Delta v}
\def\eikr{{\rm e}^{i \bk \cdot \br}}
\def\rmikr{{\rm e}^{i \vec k \cdot \vec r}}
\def\rmnikr{{\rm e}^{-i \vec k \cdot \vec r}}
\def\eikx{{\rm e}^{i \bk \cdot \bx}}
\def\muo{\mu_0}
\def\epo{\epsilon_0}
\def\rme{{\rm e}}
\def\dnu{\Delta \nu_D}
\def\rAA{\AA \enskip}
\def\halp{{\rm H_\alpha}}
\def\hbet{{\rm H_\beta}}
\def\N#1{{N({\rm #1})}}
\def\ekz{\rme^{i ( k_z z - \omega t )}}
\def\cm#1{\, {\rm cm^{#1}}}
\def\bcm#1{\; {\rm \mathbf{cm^{#1}}}}
\def\rmK{\; \rm K}
\def\rmm#1{\; {\rm #1}}
\def\Nperp{N_\perp}
\def\spkr#1{{\big \bf {#1}}}
\def\inA{\quad $\bullet$ \quad}
\def\inB{\quad\quad $\diamond$ \quad}
\def\inC{\quad\quad\quad $\circ$ \quad}
\def\sigfc{\sigma_{f_c}}
\def\sech{\rm sech}
\def\blck{\color[named]{Black}}
\def\blue{\color[named]{Blue}}
\def\skyblue{\color[named]{SkyBlue}}
\def\red{\color[named]{Red}}
\def\black{\color[named]{Black}}
\def\green{\color[named]{Green}}
\def\white{\color[named]{White}}
\def\yellow{\color[named]{Yellow}}
\def\purple{\color[named]{Purple}}
\def\maroon{\color[named]{Maroon}}
\def\redv{\color[named]{RedViolet}}
\def\mahog{\color[named]{Mahogany}}
\def\prb{$\frac{\delta \rho}{\rho}$}
\def\trd{\frac{2}{3}}
\def\mdn{\bar \rho}
\def\ss{v_s^2}
\def\msol{M_\odot}
\def\W#1{{W({\rm #1})}}
\def\Nav{N_a (v)}
\def\f#1{{f_{\rm #1}}}
\def\opt{$d \tau / d z \; $}
\def\btau{\bar\tau (v_{pk}) / \sigma (\bar \tau)}
\def\Ipk{I(v_{pk})/I_{c}}
\def\vrot{v_{rot}}
\def\sphr{\sqrt{R^2 + Z^2}}
\def\mbf{\mathbf}
\def\mgsqa{\, {\rm mag} / \Box ''}
\newcommand{\ialii}{Al{\sc II}}
\newcommand{\iari}{Ar{\sc I}}
\newcommand{\ini}{N{\sc I}}
\newcommand{\ioi}{O{\sc I}}
\newcommand{\iovi}{O{\sc VI}}
\newcommand{\iovii}{O{\sc VII}}
\newcommand{\ioviii}{O{\sc VIII}}
\newcommand{\ifeii}{Fe{\sc II}}
\newcommand{\ifeiii}{Fe{\sc III}}
\newcommand{\ispii}{S{\sc II}}
\newcommand{\iznii}{Zn{\sc II}}
\newcommand{\icaii}{Ca{\sc II}}
\newcommand{\inai}{Na{\sc I}}
\newcommand{\icii}{C{\sc II}}
\newcommand{\iciv}{C{\sc IV}}
\newcommand{\ihii}{H{\sc II}}
\newcommand{\ihi}{H{\sc I}}
\newcommand{\isiii}{Si{\sc II}}
\newcommand{\isiiv}{Si{\sc IV}}
\newcommand{\ici}{C{\sc I}}
\newcommand{\dcost}{d \negmedspace \cos\theta}
\newcommand{\mnvhi}{n_{\rm H {\sc I}}}
\newcommand{\nvhi}{$n_{\rm H {\sc I}}$}
\newcommand{\rem}{{\rm e}}
\def\otpr#1#2{{|#1><#2|}}
\def\avgq#1{<\negmedspace#1\negmedspace>}
\newcommand{\negm}{\negmedspace}
\newcommand{\dfhf}{^2\negm D_{\frac{5}{2}}}
\newcommand{\negt}{\negthinspace}
\newcommand{\nalph}{N/$\alpha$}
\newcommand{\alphh}{$\alpha$/H}
\newcommand{\nhi}{$N_{\rm HI}$}
\newcommand{\mnhi}{N_{\rm HI}}

\renewcommand{\descriptionlabel}[1]%
    {\hspace{\labelsep}\textsf{#1}}

\newenvironment{Aitemize}{%
 \renewcommand{\labelitemi}{$\bullet \;$}%
	\begin{itemize}}{\end{itemize}}

\newenvironment{Renumerate}{%
 \renewcommand{\theenumi}{\Roman{enumi}}
 \renewcommand{\labelenumi}{\theenumi)}
	\begin{enumerate}}{\end{enumerate}}

\newenvironment{Aenumerate}{%
 \renewcommand{\theenumi}{\Alph{enumi}}
 \renewcommand{\labelenumi}{\theenumi.}
	\begin{enumerate}}{\end{enumerate}}

\newenvironment{dmnd}{%
 \renewcommand{\labelitemi}{$\diamond \quad$}%
	\begin{itemize}}{\end{itemize}}

\newenvironment{Penumerate}{%
 \renewcommand{\labelenumi}{(\theenumi)}
	\begin{enumerate}}{\end{enumerate}}



\newcommand{\nnn}{ccd\#\#\#}
 
\special{papersize=8.5in,11in}

\renewcommand{\theenumi}{\Roman{enumi}}

\begin{document}
%\pagestyle{empty}
 

\begin{center}
{{\Huge \bf WFCCD Multi-Slit Reduction [v1.0]}}
\end{center}

\begin{enumerate}

{\Large   \item Requirements }
   \begin{itemize}
	\item IDL v5.4 or later
	\item $>$500M RAM
	\item $>$2G hard drive space
	\item The full software package, including the {\bf djs} and 
		{\bf idlspec2d} codes
	\item Hopefully a 300MHz processor or better
	\item Slit mask information (ideally from Andy's maskgen program)
	\item Log sheets [somewhat optional]
   \end{itemize}

{\Large   \item Suggested Calibrations}
   \begin{itemize}
	\item 2 QTZ Lamps at position of science frames
	\item 1 Arc on each side of the science frames
	\item (Optional) Bias frames to check CCD performance
   \end{itemize}

{\Large  \item Initial Setup (Repeat for each night) }

  \begin{enumerate}

    \item Create a new directory for the night and enter it
    \item Create a Raw/ directory and put all the raw data in it.  
    \item Create a pro/ directory and copy the files from 
	$\$$XIDL/Spec/WFCCD/pro into it 
	(e.g.\ {\it setcrds.pro, proc\_night.pro, extract\_night.pro}).
	{\bf These files can be used to run most of the steps that follow.}
    \item Launch idl in the directory above Raw/
    \item {\bf wfccd\_strct} :: Create the wfccd structure.  
	Again, I use the {\it proc\_night.pro} file to do this.
	This structure organizes
	the entire night of data and is the most important file created.  The 
	structure is listed on the next page [v1.1].  \\
	The routine creates a few things: (1) an IDL structure in memory with
	whatever name you choose (e.g.\ night1); 
	(2) the file 'struct.fits' which is a fits
	version of the structure; (3) the file 'wfccd.list' which is an ASCII
	version of the fits file which lists the values of some but not all tags.
	To view the structure outside of IDL (or even in general), I highly 
	recommend the program 'fv' which I think stands for fitsview.  It allows
	you to examine binary fits tables. \\
         \quad Example: IDL$> \;$ {\bf wfccd\_strct}, night1, /MKDIR \\  
         \quad Time   : $<1$s per image

    \item Modify the WFCCD structure :: 
	The previous step creates the structure and takes a guess at the 
	initial values of many of the tags based on the header card info.  It
	is impossible, however, to automate all of the values for the tags and
	therefore the user must do a fair amount of this by hand.  The file
	{\it setcrds.pro} 
	is an extensive example.  Note the IDL routine {\bf x\_fmidx} converts
	the frameno of an image to the index in the structure. \\	
	The obvious tags to modify are: 
	 \begin{enumerate}
	  \item Obj :: Object name 
	  \item flg\_anly :: Include in analysis (defaulted to 1 for yes)
	  \item type :: ARC, FLT, etc.  (see the struct definition)
	  \item masknm :: Generally a two character string giving the unique def
	  \item mask\_id :: Long integer identifying a given mask.  This is the
			most important
	  \item msk\_fil :: Mask file describing the slits.  You will
		always need to set this tag.  Ideally, the 
		output from Andy's {\it maskgen} program. I tend to put these
		files in the Masks/ directory.
	 \end{enumerate}

    \item Updating the WFCCD structure :: In IDL you can modify the values
	of any of the tags.  You can then save the structure in fits form and 
	rewrite the ASCII file  with the routine {\bf write\_wfccdstr}. \\
         \quad Example: IDL$> \;$ {\bf write\_wfccdstr}, night1, FITS='name' \\
         \quad Time   : fast

    \item OV region :: By default, the package will assume 32 columns of
	overscan.  If you use less, the program will behave fine.  If
	you use more, they will be ignored.
  \end{enumerate}


\clearpage
	\begin{center}
	{\Large {\bf wfccdstr}}
	\end{center}

	{\small
	\begin{tabular}{lcl}
	  \hline
	  Tag & Type & Comment \\
	  \hline
         frame & 0     & FRAME Number \\
         flg\_anly & 0   &  Analysis flag 0=Don't Analyse \\
         Obj & ' '       & Object Name \\
         type & ' '    & ObjTyp : OBJ, STD, DRK, ZRO, FLT, ARC, MSK, ALG \\
         mask\_id & 0L    & Mask ID \\
         masknm & ' '    & Mask Name \\
         exp & 0.d        & Exposure time \\
         filter & ' '     & Filter : U,B,V,R,I, C \\
         filtpos & 0     & Filter Position \\
         grism & ' '     & Grism : BG, RG, NO \\
         aperpos & 0     & Aperture position \\
         casspos & 0.    & CASS pos angle \\
         AM &   0.     & Airmass \\
         CCD & ' '     & CCD \\
         TEL & ' '    & Telescope \\
         gain & 0.     & Gain \\
         readno & 0.     & Read Noise \\
         date & 0.0d    & Date of Obs \\
         UT & ' '    & UT \\
         RA & ' '    & RA \\
         DEC & ' '    & DEC \\
         Equinox & 0.     & EQUINOX \\
         rootpth & ' '    & Path of the Root \\
         img\_root & ' '   & Root name (usually in Raw directory) \\
         flg\_ov &  0     & OV FILE?  0=No, 1=Yes   \\
         img\_ov &  ' '   & Name of OV file (with directory) \\
         flg\_msk &  0     & Mask FILE?  0=No, 1=Yes   \\
         img\_msk &  ' '   & Name of Mask file \\
         flg\_final & 0   & Final File? 0=No \\
         img\_final & ' '  & Name of Final img \\
         nslits & 0     & Num of slits \\
         ystrt & 0L      & Column for initiating the trace \\
         msk\_fil & ' '  & Name of the Mask info file (fits) \\
         slit\_fil & ' '  & Name of the Slit info file (fits) \\
         arc\_fil & ' '   & Name of the Arc image file (fits) \\
         map\_fil & ' '   & Name of the Map image file (fits) \\
         flat\_fil & ' '   & Name of the Flat image file (fits) \\
         obj\_fil & ' '  &  Name of object structure (fits) \\
	  \hline
	\end{tabular}
	}
	
\vskip 0.2in

{\Large    \item Create bias frame (optional)}
	\begin{enumerate}
	  \item As a check you might create a bias frame.  At the moment
	the reduction process does not include the bias frame.
	  \item Run {\bf wfccd\_bias}. This routine outputs an image
	'Bias.fits' in the Bias/ directory. The routine creates and then
	deletes overscan frames for each image.\\
         \quad Example: IDL$> \;$ {\bf wfccd\_bias}, night1 \\
         \quad Example: IDL$> \;$ {\bf proc\_night}, night1, /MKBIAS \\
         \quad Time   : 3-5min \\
	  \item To check the Bias frame : {\bf xatv}, 'Bias/Bias.fits'
	\end{enumerate}

\end{enumerate}

\clearpage


\noindent {\bf {\Large \quad 0. Mask Reduction}}

  \begin{itemize}
	\item The following routines all apply to a single mask.  In general,
	this means multiple science exposures of one designed mask.  One
	could break these up too, however.
	\item I suggest you create a .pro file like {\it proc\_night.pro} 
	to guide the process.  
	It provides a convenient way of stepping through the reduction.
	\item Most of the following routines take only the WFCCD structure
	and the {\it maskid} as inputs.
  \end{itemize}

\begin{enumerate}

{\Large  \item Combine Flats }
  \begin{itemize}
	\item Generally, one has at least 2 flats per mask.  To remove CR,
	if nothing else, I combine the images.
	\item Run {\bf wfccd\_mkflat} :: This routine uses {\bf x\_addtwoflats} to
	combine 2 flats if there are only two or {\bf x\_combine} for more than
	2 flats. 
	\item Output is 'Flats/Flat\_{\it maskid}.fits' \\
         \quad Example: IDL$> \;$ {\bf wfccd\_mkflat}, night1, maskid \\
         \quad Example: IDL$> \;$ {\bf proc\_night}, night1, maskid, /MKFLAT \\
         \quad Time   : $<1$min \\

	\item Check the Flat: {\bf xatv}, 'Flats/Flat\_{\it maskid}.fits'
  \end{itemize}

{\Large  \item Create the Map}
  \begin{itemize}
	\item {\bf wfccd\_mkmap} ::
	This step produces a fits file which contains information
	relating the original data frame to one where the slits run straight
	accross the image.  The 'straightened' image has the same number of
	total rows as the original.
	\item The main routine {\bf x\_traceflat} takes the following steps: 
	  \begin{enumerate}
	    \item transposes the flat image
	    \item creates a sawtooth image (shifts and subtracts) to highlight
		slit edges
	    \item Keys on column (row) 400 to find all significant peaks in the
		sawtooth image
	    \item Uses {\bf trace\_crude} to trace the slit edges
	    \item Outputs a trace structure {\bf tracestrct}
	  \end{enumerate}
	\item The user then interactively checks the various traces with the
	routine {\bf x\_ydtortgui}. 
	\begin{itemize}
	  \item You can resize this gui
	  \item LMB : Stretches the contrast and brightness.  
	  \item RMB : Edit the ends of each trace.  The program assumes
		you have chosen the nearest edge and changes good to bad
		or vice-versa.
	  \item DELTRC : Delete a given trace altogether.  
	\end{itemize}
	I advise you extend the RHS of traces where it is sensible as there
	is generally little signal out at these edges.
	This program is one of the most user intensive of the 
	entire reduction package.  I intend to automize it entirely
	\item Finally, I run {\bf x\_fit2dtrace} to fit a 2D polynomial to 
	the traces and create a map of the transformation.  I also create the
	inverse map.
	\item Output is 'Maps/Map\_{\it maskid}.fits' and 
	'Maps/Trc\_{\it maskid}.fits.  The map files are saved as double images
	and take 67M memory. \\
         \quad Example: IDL$> \;$ {\bf wfccd\_mkmap}, night1, maskid \\
         \quad Time   : 5min to check the trace + 5min for the Map creation \\
	\item Check the Map
	 \begin{enumerate}
		\item Read in the flat: \quad 
		flat = mrdfits('Flats/Flat\_{\it maskid}.fits')
		\item Read in the map: \quad 
		map = mrdfits('Maps/Map\_{\it maskid}.fits')
		\item Rectify: \quad rflat = x\_rectify(flat, map)
		\item Examine: \quad {\bf xatv}, rflat 
	 \end{enumerate}
  \end{itemize}

{\Large  \item Create the Slit Structure}
  \begin{enumerate}
	\item {\bf wfccd\_setslits} ::
	This step creates a slit structure which describes the slits in a mask
	in greater detail than the WFCCD structure.  The user can interactively
	set the slit edges.  This is important for dealing with slit overlap.
	Here is the structure: 

	\begin{center}
	  {\Large {\bf mslitstrct} }
	\end{center}
	{\small
	\begin{tabular}{lcl}
	  \hline
	  Tag & Type & Comment \\
	  \hline
         flg\_anly &  0  & Analyse? (1=yes, 0=no) \\
         id &  0L &  ID number \\
         field &  ' ' &  Field name \\
         length &  0. & (arcsec) \\
         width &  0.  & (arcsec) \\
         pa &  0. \\
         arcpix &  0.   & Arcsec per pix \\
         xyqso &  fltarr(2) &  x offset from the QSO (arcsec) \\
         xpos &  0. &  x position of silt \\
         nobj &  0 &  Number of objects in slit \\
         ypos &  fltarr(10) &  y pos of obj relative to slit edges ($\%$) \\
         yedg &  fltarr(2) &  Slit edges relative to slit mask cen (pix) \\
         yedg\_flt &  fltarr(2) &  y-Edges  of slit on CCD (Undistorted FLAT) \\
         yedg\_orig &  fltarr(5000,2)   &  Array of slit edges in orig \\
         yedg\_sky &  fltarr(5000,2)   &  Edges of slit for sky subtraction\\
	  \hline
	\end{tabular}
	}

\vskip 0.2in
	\item Parse the slit file listed in the WFCCD
	structure under the 'msk\_fil' tag.  
	You will need to have put your mask file in the appropriate place
	beforehand (e.g.\ 'Masks/').
	The routine is {\bf wfccd\_prsslits}.
	The slit structure is created in memory.
	\item Straighten the Flat with the Map file (uses {\bf x\_rectify}).
	\item {\bf x\_setslits} ::
	Guess at the slit positions using the info in 'msk\_fil'
	\item {\bf x\_stsltgui} :: Unless NOVERIFY is set, 
	launch a gui to interactively check
	the slits.  It is important to set the silt edges
	about right.  In particular, overlapping slits leads to a real
	problem with sky subtraction and they are best dealt with by 
	identifying the slit edges properly.
	  \begin{itemize}
	    \item You can resize the display screen
	    \item I would examine the LHS of the image where there is more flux
	    \item The slit width is always preserved
	    \item Red (blue) lines indicate the top (bottom) of a slit
	    \item z (Z) zooms in (out)
	    \item LMB sets contrast/brightness
	    \item s (S) shifts all slits below the selected one
		(inclusive) and keys on top (bottom) of slit
	    \item b (t) shifts one slit and keys on bottom (top) of slit
	  \end{itemize}
	\item Map the slits into the original frame using {\bf x\_origslit}.
	The slit edge traces are written into 'yedg\_orig' in the slit structure.
	\item Output the Slit structure to 'Slits/Slits\_{\it mask\_id}.fits' \\
         \quad Example: IDL$> \;$ {\bf wfccd\_setslits}, night1, maskid \\
         \quad Time   : $<5$min to check the slits \\
  \end{enumerate}

{\Large  \item Normalize the Flat}
  \begin{itemize}
	\item {\bf wfccd\_normflat} :: Collapses (median) the slits in the flat
	using the slit edges from the previous step.  Fits a high order
	bspline to the final result (with 5sig rejection).
	\item The main driver is {\bf x\_normflat}.
	\item The output is 'Flats/FlatN\_{\it mask\_id}.fits'. The 
	WFCCD structure is updated to reflect the new flat.  At this point
	the file 'Flats/Flat\_{\it mask\_id}.fits' is unnecessary (33M). 
	I'd check the normalized flat before ever deleting it, however.\\
         \quad Example: IDL$> \;$ {\bf wfccd\_normflat}, night1, maskid \\
         \quad Time   : $<3$min \\
  \end{itemize}


{\Large  \item Create the Wavelength Image}
  \begin{enumerate}
	\item This sequence of steps (i) processes the arc images, (ii) creates
	a wavelength solution for the central row of each slit in the
	straightened frame; and (iii) finds the wavelength solution for each
	pixel and maps back to the original frame.
	\item {\bf wfccd\_procarc} :: Process each Arc image. OV subtract, 
	flatten, straighten the slits.  Output is 
	'Arcs/ArcR\_\nnn.fits' where \nnn\ corresponds
	to the orginal ccd frame number.  \\
         \quad Example: IDL$> \;$ {\bf wfccd\_procarc}, night1, maskid \\
         \quad Time   : $<1$min \\
	\item {\bf wfccd\_allarcsol} :: Find the wavelength solution for the
	center of each slit to be analysed (alignment star boxes are skipped).
	This routine drives the program {\bf wfccd\_arcsol} which: 
	\begin{itemize}
	  \item Medians the 5 central rows of each slit
	  \item Guesses the initial wavelength solution based on the
		slit position
	  \item Calls {\bf x\_autoid} which is a 
	$\chi^2$ minimization routine that automatically finds 
	a wavelength solution
	given an initial guess, an arc spectrum, and a line list.  By
	default we use the linelist 'wfccdB\_HeNe.lst' given in 
	$\$$XIDL/Spec/Arcs/Lists
	  \item Output is an Arc structure {\bf wfccdarcstr} which is output
	as a fits file 'Arcs/ArcS\_\nnn.fits'.  It has
	the wavlength solution, arc spectrum, and a wavelength array 
	for the center of each slit in the straightened frame.
         \quad Example: IDL$> \;$ {\bf wfccd\_allarcsol}, night1, maskid\\
         \quad Time   : $\approx 10$s/slit \\
	\end{itemize}

	\item {\bf wfccd\_arcimg} :: Translate the wavelength solution from
	the center rows to the entire slit.  
	This is the only procedure which is exposure sensitive.
	The main driver is {\bf x\_arcimage}:
	\begin{itemize} 
	  \item Identifies all 3$\sigma$ arc lines and traces
	them across the slit.  
	  \item It then calculates a unique wavelength solution
	for each row.  
	\end{itemize}
	Finally, the program maps the wavelength solution
	back into the original frame. The routine is called once per exposure
	for a given mask.  It matches each exposure with the nearest Arc
	frame by considering the UT of the images.  The output is an Arc
	image 'Arcs/ArcI\_\nnn.fits'. \\
         \quad Example: IDL$> \;$ {\bf wfccd\_arcimg}, night1, maskid, expsr \\
         \quad Time   : $\approx 1$min/exposure \\

	\item {\bf wfccd\_chkarc} :: Plots the arc line solutions from each
	slit in a simple gui.  Shows the RMS solution (\AA) and the red lines
	identify key arc lines.  If something isn't right, you are in trouble!\\
         \quad Example: IDL$> \;$ {\bf wfccd\_chkarc}, night1, maskid, expsr \\
         \quad Time   : fast \\

	\item {\bf wfccd\_cleanarc} :: Zeros out bad wavelength values.  These
	should only occur when two slits are too close or overlapping.  This
	procedure helps with the sky subtraction and extraction.\\
         \quad Example: IDL$> \;$ {\bf wfccd\_cleanarc}, night1, maskid, expsr \\
         \quad Time   : $<1$min \\

  \end{enumerate}

{\Large  \item Process the Science Images}
  \begin{itemize}
	\item {\bf wfccd\_procobj} :: OV subtract, flatten and staple in 
	the wavelength image.
	\item Produces one final fits image: 'Final/f\_\nnn.fits'.
	This file has Flux, Var, Wave in the first 3 data units (50Mb). \\
    	\quad Example: IDL$> \;$ {\bf wfccd\_procobj}, night1, maskid \\
  	\quad Time   : fast \\
  \end{itemize}

{\Large  \item CR Flagging}
  \begin{itemize}
	\item {\bf wfccd\_objcr} :: Identifies CR in the object frames by
	comparing two exposures.  Currently can only take two images at a time
	but this is easily updated.  A CR must have a flux $>3 \times$ the
	pixel in the other frame and a flux $> 200$.\\
    	\quad Example: IDL$> \;$ {\bf wfccd\_objcr}, night1, maskid \\
  	\quad Time   : $< 2$min \\
  \end{itemize}

{\Large  \item Clean Up}
  \begin{itemize}
	\item {\bf wfccd\_cleanup} :: Removes many of the unnecessary files that
	the package creates (e.g.\ the unnormalized flat).  Don't run this
	unless you are quite confident your processing is complete.\\
    	\quad Example: IDL$> \;$ {\bf wfccd\_cleanup}, night1, maskid \\
  	\quad Time   : fast \\
  \end{itemize}


\end{enumerate}

\clearpage


\noindent {\Large \quad 0. Mask Extraction}

  \begin{itemize}
	\item The following routines all apply to a single exposure  
	\item I suggest you create a .pro file like {\it extrct\_msk.pro} 
	to guide the process.  It provides a convenient way 
	of stepping through the reduction.
	\item Most of the following routines take the WFCCD structure, 
	the {\it maskid}, and the exposure number as inputs.
  \end{itemize}

\begin{enumerate}

{\Large  \item Create Object Structure}

  \begin{enumerate}
	\item {\bf wfccd\_mkobjstr} :: Creates the object structure which
	stores the spectra of all objects idenitified in the mask. 
	\item Options: 
	  \begin{itemize}
		\item /NOSKY :: Will not modify the region for sky subtraction
		\item /NOCHK :: Will not run {\bf x\_setobjgui}
		\item /NOSLIT :: Will not modify the Slit structure
		\item /NOOBJ :: Will not modify the Obj structure
		\item SKYOFF :: Offset of sky edge from slit edge (default=2pix)
	  \end{itemize}

	\begin{center}
	  {\Large {\bf specobjstrct} }
	\end{center}
	{\small
	\begin{tabular}{lcl}
	  \hline
	  Tag & Type & Comment \\
	  \hline
         field &  ' ' &  Name of field \\
         slit\_id &  0L \\
         obj\_id &  ' ' & ID value (a=primary, b-z=serendip, x=NG) \\
         flg\_anly &  0 &  0=No analysis\\ 
         exp &  0. & \\
         xcen &  0L &  Column where obj was id\\
         ycen &  0. & \\
         flg\_aper &  0 &  0=boxcar\\
         aper &  fltarr(2) &  Widths of aperture, 0/1 = bottom/top (pixels)\\
         skyrms &  0. &  RMS of sky fit\\
         trace &  fltarr(5000) &\\
         npix &  0L & \\
         wave &  fltarr(5000) & \\
         fx &  fltarr(5000) & \\
         var &  fltarr(5000) &    $<=0$  rejected pix\\
         flg\_flux & 0 &    0=f$_\lambda$, 1=f$_\nu$\\
         flux &  fltarr(5000) &    Fluxed data\\
         sig &  fltarr(5000) &    Err in fluxed data\\
         date &  0.0d \\
         UT &  ' ' \\
         img\_fil &  ' ' \\
         slit\_fil &  ' ' \\
         instr\_strct &  ' ' & e.g. wfccdstr fits file\\
	  \hline
	\end{tabular}
	}

	\item The program first runs {\bf x\_fndslitobj} which automatically
	looks for the main science object (as defined in the Mask output
	file) and any serendipitous objects.
	\item The user should then run {\bf x\_setobjgui} to confirm the 
	object locations and set the slit edges to their 'final' locations
	for sky subtraction. Rough centroid for the objects is good enough.
	{\it In general, sky subtraction works best on regions $\approx 25$ pix
	wide.}  Beware of objects near the slit edge.  Try to extend the sky
	regions beyond them.
	\begin{itemize}
	  \item xatv-like display (use the Help window)
	  \item Green (blue) lines trace science (serendipitous) objects
	  \item Red crosses identifies the guess from the slit info
	  \item 'm' moves the closest object to the cursor position
	  \item 'd' deletes closest obj
	  \item '{}[]' = pan up,down,left,right
	  \item b (t) shifts the slit edge at the bottom (top) of slit
	\end{itemize}
	\item Output is 'Extract/Obj\_{\it maskid}.fits' and the Slit
	structure is updated.\\
         \quad Example: IDL$> \;$ {\bf wfccd\_mkobjstr}, night1, maskid, expsr \\
         \quad Time   : 3min/exposure for slit+obj checking \\
	\item {\bf wfccd\_editobjstr} :: Edit the Obj structure. Options: 
	/NOOBJ, /NOSLIT.  \\
         \quad Example: IDL$> \;$ {\bf wfccd\_editobjstr}, night1, maskid, expsr \\
         \quad Time   : fast\\
  \end{enumerate}

{\Large  \item Sky Subtraction}
  \begin{itemize}
	\item {\bf wfccd\_skysub} ::
	This procedure subtracts the sky from each slit, one by one.
	The main driver is {\bf x\_subskyslit} which does a LEGENDRE fit
	(the better of 2nd or 3rd order provided enough pixels) 
	to each column after masking out the objects
	in the slit. It works on individual exposures.  Currently, my routines
	do not handle objects right at the slit edge very well.
	\item Output is the sky subtracted 2D frame 
	'Extract/F\_\nnn.fits' with the appended Sky image and
	the rms of the fit along each column for each slit (35M).\\
         \quad Example: IDL$> \;$ {\bf wfccd\_skysub}, night1, maskid, exposure \\
         \quad Time   : 20-30s/slit (roughly 10-15min) \\
  \end{itemize}

{\Large  \item Extraction}
  \begin{enumerate}
	\item {\bf wfccd\_extobjbox} ::
	This procedure extracts the objects from the slits one by one.
	It works on individual exposures seperately.
	The main driver is {\bf x\_extobjbox} which 
	\begin{enumerate}
	  \item	Traces the object with {\bf trace\_crude}
	  \item Smashes the object to determine its profile
	  \item Determines an aperture which includes 95$\%$ of the flux
	  \item Identify pixels included in the aperture and what fraction
	  \item Flags CR's by comparing the flux in a given column against
		the object profile.  CR events have variance set to $-1$.
	  \item Calls {\bf x\_rebin2dspec} which sums the flux into a 
		1D array with starting wavelength
		3200\AA\ and $\delv = 100$km/s pixels
	\end{enumerate}
	\item  All of the output is written to the object structure \\
         \quad Example: IDL$> \;$ {\bf wfccd\_extobjbox}, night1, maskid, expsr \\
         \quad Time   : $<5$min \\
	\item Run {\bf wfccd\_objsumm} to get a summary of the Fspec file.
  \end{enumerate}

{\Large  \item Check/Edit the Individual Spectra}
  \begin{itemize}
	\item {\bf wfccd\_chkspec} :: This routine plots up all of the
	objects in the object structure for a quick glance.
	It is a very simple gui.  Bad extractions are easily identified
	and can be removed by adjusting the object structure 'by hand'.\\
         \quad Example: IDL$> \;$ {\bf wfccd\_chkspec}, night1, maskid, expsr \\
         \quad Time   : fast \\
	\item {\bf wfccd\_pltobj} :: This routine plots the spectrum and the
	2D image together (very nice).  You can examine the 'reality' of 
	various emission and absorption lines. \\
         \quad Example: IDL$> \;$ {\bf wfccd\_pltobj}, night1, maskid, 
		expsr, [objnm] \\
         \quad Time   : 5s \\
	\item {\bf wfccd\_editspec} :: This routine allows the user to set the
	variance of regions of spectra to 0 to eliminate them from further
	analysis. \\
         \quad Example: IDL$> \;$ {\bf wfccd\_editspec}, night1, maskid, expsr \\
         \quad Time   : fast \\
  \end{itemize}

{\Large  \item Flux the Spectra}
  \begin{itemize}
	\item {\bf wfccd\_fluxspec} :: This routine plots up all of the
	objects in the object structure for a quick glance.
	It is a very simple gui.  Bad extractions are easily identified
	and can be removed by adjusting the object structure 'by hand'.\\
         \quad Example: IDL$> \;$ {\bf wfccd\_fluxspec}, night1, maskid, [expsr] \\
         \quad Time   : fast \\
  \end{itemize}

{\Large  \item Combine the Spectra}
  \begin{itemize}
	\item {\bf wfccd\_combspec} :: This routine combines multiple object
	structures (primarily their spectra) into one Final structure. 
	For objects with multiple
	spectra, the spectra are matched to the same flux and summed.  At
	the moment this routine only handles two spectra at a time.  \\
	\item Output :: A structure containg all of the fluxed, coadded spectra
	with name : {\it 'Extract/Fspec\_maskid.fits'}.  
         \quad Example: IDL$> \;$ {\bf wfccd\_combspec}, night1, maskid \\
         \quad Time   : fast \\
	\item Run {\bf wfccd\_fspecsumm} to get a summary of the Fspec file.
	\item Run {\bf wfccd\_chkfspec} to view the Final spectra \\
         \quad Example: IDL$> \;$ {\bf wfccd\_chkfspec}, 'Extract/Fspec\_maskid.fits'\\
         \quad Time   : fast \\
  \end{itemize}

{\Large  \item Solve for Redshifts}
  \begin{itemize}
	\item {\bf wfccd\_zfind} :: This routine drives the SDSS {\bf zfind} routines
	and uses the SDSS eigenspectra (rebinned to WFCCD pixel scale) to
	automate redshift identification.  It is a $\chi^2$ minimization routine
	and it works rather well.  Like all $\chi^2$ routines, however, it tends
	to fail when there are particularly bad regions of data.  The ZANS
	structure in the Fspec structure is updated. \\
         \quad Example: IDL$> \;$ {\bf wfccd\_zfind}, night1, maskid \\
         \quad Time   : Long, $\approx 30$s per obj \\
	\item Run {\bf wfccd\_chkfspec} with the /ZFIND switch to 
		view the Final spectra + \\
         \quad Example: IDL$> \;$ {\bf wfccd\_chkfspec}, 
		'Extract/Fspec\_maskid.fits', /zfind \\
         \quad Time   : fast \\
  \end{itemize}

\end{enumerate}

\end{document}
	

